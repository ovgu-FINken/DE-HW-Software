\hypertarget{index_intro_sec}{}\section{Introduction}\label{index_intro_sec}
Our project is to develop hardware for the F\+I\+Nken 3 robots.

We have designed the P\+C\+Bs where the reference link is given below for the further details \href{https://github.com/ovgu-FINken/DE-HW-Hardware/wiki}{\tt Hardware link}

This document mainly focused on the software part.

Currently the F\+I\+Nken I\+II is the newest generation of our copters with more power,better communication and more sensors

{\bfseries Specifications\+:}


\begin{DoxyEnumerate}
\item Optical flow sensor with sonar ranging towards ground (Optional\+: Infrared distance sensor towards ground)
\item Tower of 4 sonar ranging sensor
\item 802.\+15.\+4 based phase-\/difference ranging between copters and anchors
\item Paparazzi\+U\+AV board (Lisa/M 2.\+1) -\/ with floating point support
\item 802.\+15.\+4 Communication Module to communicate to ground station and between copters
\end{DoxyEnumerate}\hypertarget{index_Software_sec}{}\section{Software Development}\label{index_Software_sec}


\hypertarget{index_comp_sec}{}\subsection{Components used in our project}\label{index_comp_sec}




List of components is provided below\+:


\begin{DoxyEnumerate}
\item Optical flow sensor with sonar ranging towards ground
\item 8 Sonars ranging sensors
\item Paparazzi\+U\+AV board
\item R\+GB L\+E\+Ds
\item Illuminating Sensor
\item S\+T\+M32\+F44\+R\+E\+T6 Micro controller
\end{DoxyEnumerate}\hypertarget{index_about}{}\section{Programming language and installation guide for using in Linux}\label{index_about}
We programmed in Clion using mbed libraries in C++ language. For using mbed libraries, mbed cli needs to install for exporting into I\+DE.

Below are the procedure as follows\+:


\begin{DoxyEnumerate}
\item Install Python v2.\+7
\item Install Git and Mercurial
\item Install G\+NU A\+RM Embedded Toolchain Step0\+: Remove previous version of toolchains, if needed \char`\"{}sudo apt-\/get remove gcc-\/arm-\/none-\/eabi\char`\"{} \char`\"{}sudo apt-\/get remove g++-\/arm-\/linux-\/gnueabi\char`\"{} \char`\"{}sudo apt-\/get remove binutils-\/arm-\/none-\/eabi\char`\"{} Step1\+: Inside Ubuntu, open a terminal and input \char`\"{}sudo add-\/apt-\/repository ppa\+:team-\/gcc-\/arm-\/embedded/ppa\char`\"{}
\end{DoxyEnumerate}

Step2\+: Continue to input \char`\"{}sudo apt-\/get update\char`\"{}

Step3\+: Continue to input to install toolchain \char`\"{}sudo apt-\/get install gcc-\/arm-\/embedded\char`\"{}
\begin{DoxyEnumerate}
\item Install mbed-\/cli \char`\"{}sudo pip install mbed-\/cli\char`\"{}
\item Get the program
\item Select tollchain in mbed-\/cli. Run in folder my\+\_\+program/mbed\+\_\+os \char`\"{}mbed config -\/-\/global G\+C\+C\+\_\+\+A\+R\+M\+\_\+\+P\+A\+T\+H /usr/bin/\char`\"{}
\item Compile the programm. Run in my\+\_\+program folder \char`\"{}mbed compile -\/t G\+C\+C\+\_\+\+A\+R\+M -\/m N\+U\+C\+L\+E\+O\+\_\+\+F411\+R\+E\char`\"{}
\item Flash the microcontroller
\end{DoxyEnumerate}

Use another terminal for running the code (for convinence) for gdb terminal

(gdb) target extended-\/remote /dev/tty\+A\+C\+M0 Remote debugging using /dev/tty\+A\+C\+M0 If A\+C\+M0 doesnt work need to try for A\+C\+M1 and A\+C\+M2 (gdb) monitor swdp\+\_\+scan....For supplying power Target voltage\+: 3.\+4V

Available Targets\+: No. Att Driver 1 S\+T\+M32\+F4xx

(gdb) attach 1 load the .elf file run if you want to break the code and check we can do the particular check

Open gtk terminal where one can see the output, set the port (usually A\+C\+M0) and change from A\+S\+C\+II to hex to see the output in hex format (in bytes)

{\bfseries Communication with Paparazzi}

The message structures is as follows for both receive and send for paparazzi\+:


\begin{DoxyEnumerate}
\item Number of submessages -\/ 1 byte
\item Submessages (details below) -\/ N $\ast$ (1 + 1 + 1 + X) bytes
\item Checksum -\/ 1 byte
\end{DoxyEnumerate}

Submessages structure is as follows\+:


\begin{DoxyEnumerate}
\item Type of sensor -\/ 1 byte
\item Id -\/ 1 byte
\item Length of data -\/ 1 byte
\item Data -\/ N bytes (described in length)
\end{DoxyEnumerate}

The data length for each component is given below

{\bfseries Message to Paparazzi}

\tabulinesep=1mm
\begin{longtabu} spread 0pt [c]{*{2}{|X[-1]}|}
\hline
\rowcolor{\tableheadbgcolor}\textbf{ Sensor }&\PBS\centering \textbf{ Data  }\\\cline{1-2}
\endfirsthead
\hline
\endfoot
\hline
\rowcolor{\tableheadbgcolor}\textbf{ Sensor }&\PBS\centering \textbf{ Data  }\\\cline{1-2}
\endhead
\hyperlink{class_sonar}{Sonar} &\PBS\centering 2 \\\cline{1-2}
I\+Rsensor &\PBS\centering 4 \\\cline{1-2}
L\+E\+Dstrip &\PBS\centering 0 \\\cline{1-2}
\end{longtabu}
{\bfseries Message from Paparazzi}

Currently only \hyperlink{class_l_e_d_strip}{L\+E\+D\+Strip} listen to messages from Paparazzi (by default).

\tabulinesep=1mm
\begin{longtabu} spread 0pt [c]{*{2}{|X[-1]}|}
\hline
\rowcolor{\tableheadbgcolor}\textbf{ Sensor }&\PBS\centering \textbf{ Data  }\\\cline{1-2}
\endfirsthead
\hline
\endfoot
\hline
\rowcolor{\tableheadbgcolor}\textbf{ Sensor }&\PBS\centering \textbf{ Data  }\\\cline{1-2}
\endhead
\hyperlink{class_l_e_d_strip}{L\+E\+D\+Strip} &\PBS\centering 4 bytes$\ast$\+No.of L\+E\+Ds \\\cline{1-2}
\end{longtabu}
4 bytes per L\+ED\+: id, red channel, green channel, blue channel. 